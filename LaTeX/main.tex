\documentclass{article}
\usepackage[utf8]{inputenc}
\usepackage[T1]{fontenc}

\title{Projet Développement Web}
\author{HAUF Jocelyn \and ROME Sylvain}
\date{Mai 2021}


\begin{document}

\maketitle

\newpage

\tableofcontents

\newpage

\section{Introduction}


Afin de réaliser ce projet de développement web, nous avons choisi de faire un binôme dans le but de nous partager les tâches et de mettre en commun nos différentes connaissances.\\
Pour commencer nous avons réfléchi au type de site que nous voulions créer, il fallait qu'il corresponde aux attentes du sujet, mais aussi que son code soit à notre portée. De ce fait le choix fût de créer un site à but informatif avec la possibilité d'interaction avec l'utilisateur.\\
Par la suite, il nous fallait trouver une thématique intéressante et correspondant au caractère informatif du site, ainsi nous nous sommes mit d'accord sur un thème qui nous plaît à tous deux et sur lequel nous avions quelques connaissances, qui est la \textbf{Formule 1}.\\

\section{Style}

Pour le style, nous avons choisi d'avoir un fond sombre afin d'avoir un rendu non agressif pour les yeux et de pouvoir travailler dans de bonnes conditions tout le long de notre programmation.\\
Aussi, nous avons récupéré une police inspirée de l'originale de la Formule 1.\\
De plus, la grande utilisation des propriétés de pourcentage proposées par \emph{css} nous permet d'avoir une fenêtre qui s'adapte lorsque l'utilisateur la rapetisse.\\
\\
Nous avons séparés les différents fichiers \emph{CSS} en fonction de leur page liées afin de se repérer facilement lors des modifications.\\

\section{Pages}
\subsection{Accueil}
La page d'accueil était initialement exclusivement écrite en \emph{HTML} (combinée avec un \emph{CSS}) afin de nous donner un aperçu global et avoir une sorte de maquette du résultat attendu, nous avons donc mit un menu de navigation qui pourrait être mis à jour ultérieurement en cas de nouveaux liens, ainsi qu'une zone de présentation pour donner les informations essentielles à l'utilisateur.\\
Aussi nous avons trouvé intéressant d'ajouter un système de défilement d'image, comme nous pouvons retrouver sur de nombreux site actuels, dans le but d'apporter un effet dynamique.\\
Cela avec l'usage de \emph{JavaScript} qui nous a aussi permis de mettre en oeuvre l'ouverture d'un menu par le biais d'un bouton présent sur la page, toujours pour accentuer cet aspect dynamique.

\subsection{Nextrace}
Cette page à été créée dans le but d'afficher un décompte avant la prochaine course de \textbf{Formule 1}, pour ce faire nous avons utilisé \emph{JavaScript} afin d'avoir un écoulement de secondes sans devoir obliger un rafraîchissement de la page.
Nous avons préféré utiliser un format \emph{php} afin que la page se mette à jour sans le besoin de remettre à niveau le code. Pour cela, nous avons créé un fichier \emph{texte} dans lequel sont entrés plusieurs indications sur les prochaines courses séparées par un caractère particulier afin d'être lues par la page, en outre, nous avons insérer un système de commentaire dans le fichier. Cela nous a permis d'avoir une première approche avec les serveurs hors faculté (MAMP) et de mettre en pratique les premières méthodes vues en cours à l'aide des fonctions de traitement de fichier (\emph{fopen()}).Mais aussi de découvrir des fonctions comme \emph{time()} qui permettent de récupérer la date actuelle ou d'autres. 
\subsection{Actu}

\subsection{Blog}

\section{Conclusion}

\section{Ressource}

Site des polices :\\ https://imjustcreative.com/download-formula-1-fonts/2018/07/16

\end{document}
